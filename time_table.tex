\documentclass{article}
\usepackage[utf8]{inputenc}
\usepackage{booktabs}

\title{Die Geheimnisse von der Berechnung von Stundenplänen}
\date{2017-01-20}
\author{Bernd Schröder}

\begin{document}
\pagenumbering{gobble}
\maketitle
\newpage
\pagenumbering{arabic}

\section{Problembeschreibung}
Basierend auf der Hauptbedingung und den Randbedingungen ist ein Algorithmus zu entwickeln, der es ermöglicht, Lösungen für einen Stundenplan vorzuschlagen, die sich der Hauptbedingung und den Randbedingungen annähern.

\subsection{Lösungsansatz}
Definiert man die Klasse, die Lehrkraft sowie den Raum als Ressourcen, so müssen zu einem bestimmten Zeitpunkt alle Ressourcen exklusiv zu einer Unterrichtsstunde zur Verfügung stehen.



\subsubsection{xxx}
Randbedingungen

Alle Randbedingungen sind in der Regel nicht zu erfüllen. Deshalb ist es entscheidend, die Menge der Lösungen nach der Abhängigkeit der Priorität der Randbedingungen zu sortieren.

    Grundschulen dürfen keine unbelegten Stunden im Unterrichtsplan haben
    Hauptfächer sollten am Anfang des Tages liegen, praktische Fächer am Ende des Tages
    Es gibt Unterrichtseinheiten, die zu Doppelstunden gekoppelt sind
    Für einige Unterrichtsstunden steht nur ein exklusiver Raum zur Verfügung
    Der Stundenplan der Lehrer sollte keine Freistunden aufweisen


More text.

\begin{table}[h!]
  \centering
  \caption{Caption for the table.}
  \label{tab:table1}
  \begin{tabular}{l|c||r}
    1 & 2 & 3\\
    \hline
    a & b & c\\
  \end{tabular}
\end{table}

\begin{table}[h!]
  \centering
  \caption{Caption for the table.}
  \label{tab:table1}
  \begin{tabular}{ccc}
    \toprule
    Some & actual & content\\
    \midrule
    prettifies & the & content\\
    as & well & as\\
    using & the & booktabs package\\
    \bottomrule
  \end{tabular}
\end{table}



\end{document}
